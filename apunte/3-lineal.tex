\section{Sesión 1: Introducción}

Consideremos el siguiente problema ilustrativo:

\begin{tcolorbox}[title={Problema introductorio}]
    Sobre una línea recta se desplazan dos móviles $A$ y $B$ siendo observados por un observador $O$ ubicado entre ellos. El móvil $A$ inicia su viaje a 1 [m] de distancia hacia la derecha de $O$, y desplazándose a 1 [m/s] hacia la derecha, mientras que el movil $B$ inicia su viaje a $1$ [m] hacia la izquierda, desplazándose a 2 [m/s] hacia la derecha. Encuentre el instante y ubicación en los cuales los móviles se encuentran.
\end{tcolorbox}

Para plantear matemáticamente este problema, podemos considerar las distancias hacia la derecha como \textit{positivas} y hacia la izquierda como \textit{negativas}. Estableciendo así lo que en física se conoce como un \textbf{sistema de referencia}. De este modo, las ecuaciones de movimiento de los dos móviles se pueden escribir como sigue:
\begin{align*}
    x_A(t) &= 1 + t \\
    x_B(t) &= -1 + 2t \\
\end{align*}
Buscamos entonces $t^*$ tal que $x_A(t^*) = x_B(t^*) := x^*$. Estas dos incógnitas $(x^*, t^*)$ son entonces soluciones al siguiente \textbf{sistema de ecuaciones}
\begin{align*}
    x - t &= 1 \\
    x - 2t &= -1 \\
\end{align*}

Entonces, podemos definir lo siguiente:

\begin{definition}[$\R^2$, igualdad, suma, ponderación, producto matriz-vector]
Los vectores de 2 dimensiones los definimos como sigue:
\begin{itemize}
\item Definimos $\R^2$ como el conjunto de los pares ordenados de números, que en adelante denotaremos verticalmente como:
$$  \begin{pmatrix} x \\ y \end{pmatrix} \in \R^2 $$
\item Se define la igualdad entre vectores como sigue:
$$  \begin{pmatrix} x \\ y \end{pmatrix} = \begin{pmatrix} z \\ w \end{pmatrix} \iff (x = z) \wedge (y = w) $$
\item Se define la suma entre vectores como:
$$  \begin{pmatrix} x \\ y \end{pmatrix} \pm  \begin{pmatrix} z \\ w \end{pmatrix} = \begin{pmatrix} x \pm z \\ y \pm w \end{pmatrix} $$
\item También se define el producto escalar o \textit{ponderación} de vectores como:
$$  \lambda  \begin{pmatrix} x \\ y \end{pmatrix} = \begin{pmatrix} \lambda x \\ \lambda y \end{pmatrix}$$
\item Y finalmente, se define el \textit{producto matriz-vector} como:
$$  \begin{pmatrix} a & b \\ c & d \end{pmatrix} \begin{pmatrix} x \\ y \end{pmatrix} = x \begin{pmatrix} a \\ c \end{pmatrix} + y \begin{pmatrix} b \\ d \end{pmatrix} $$
\end{itemize}
\end{definition}

Con estas definiciones, el sistema de ecuaciones antes mencionado se puede escribir como el problema de encontrar $(x, t)$ tales que:

$$  \begin{pmatrix} 1 & -1 \\ 1 & -2  \end{pmatrix} \begin{pmatrix} x \\ t \end{pmatrix} = \begin{pmatrix} 1 \\ -1  \end{pmatrix}$$

Y esto se puede entender como un problema simple de \textit{text} una función. Es decir, para $f: A \to B$ e $y \in B$, encontrar $x \in A$ tal que $f(x) = y$. El vector al lado derecho juega el rol de $y$, y la matriz juega el rol de la función $f$. Esto nose lleva entonces a preguntarnos por las propiedades que puede tener la función $f(x) = Ax$ siendo $A$ una matriz y $x$ un vector. Esto será tema de la siguiente sesión.

\subsection{Problemas teóricos}

Antes de pasar al estudio de las propiedades del producto matriz-vector, haremos las definiciones de una manera más general. Para ello, primero comprenda las siguientes definiciones.

\begin{definition}[$\R^n$, igualdad, suma, ponderación, producto matriz-vector]
Los vectores en $n$ dimensiones los definimos como sigue:
\begin{itemize}
\item Definimos $\R^n$ como el conjunto de las $n$-tuplas ordenadas de números, las cuales denotaremos como:
$$ x = \begin{pmatrix} x_1 \\ x_2 \\ \vdots \\ x_n \end{pmatrix} \in  \R^n$$
\item Definimos la igualdad entre vectores en $\R^n$ como:
$$ x = y \iff  \forall i \in \{1, \dots, n\}: x_i = y_i $$
\item Definimos la suma entre vectores de $\R^n$ como:
$$ (x \pm y)_i = x_i \pm y_i $$
\item Definimos el producto escalar o \textit{ponderación} para vectores de $\R^n$ como:
$$ (\lambda x)_i = \lambda x_i $$
\item Las matrices las entendemos como \textit{bloques} de números reales, que en general no serán cuadrados sino rectangulares. Estos bloques los denotaremos como:
$$ A = \begin{pmatrix} a_{11} & \dots & a_{1n} \\ \vdots & \ddots & \vdots \\ a_{m1} & \dots & a_{mn} \end{pmatrix} \in  \R^{m \times n} $$
Diremos que la matriz $A$ tiene $n$ columnas y $m$ filas. Las cuales frecuentemente querremos aislar y considerar como vectores en su propio mérito (ver la definición dada anteriormente para el producto matriz-vector). Por lo tanto, definiremos los vectores columna $A_{\bullet j} \in  \R^m$ para $j \in \{1, \dots, n\}$ por $(A_{\bullet j})_i = a_{ij}$. Además, definimos los vectores fila $A_{i \bullet} \in  \R^n$ para $i \in \{1, \dots, m\}$ por $(A_{i \bullet})_j = a_{ij}$. En otras palabras:
$$ A = \begin{pmatrix} & & \\ A_{\bullet  1} & \dots & A_{\bullet n} \\ & & \end{pmatrix} = \begin{pmatrix} & A_{1  \bullet} & \\ & \vdots & \\ & A_{m \bullet} & \end{pmatrix} $$
\item Con estas nociones, se define el producto matriz-vector para un vector $x \in  \R^n$ y una matriz $A \in  \R^{m \times n}$ por
$$ Ax = x_1 A_{\bullet  1} + \dots + x_n A_{\bullet n} = \sum_{j = 1}^n x_j A_{\bullet j} \in  \R^m$$
Es decir, $A \in  \R^{m \times n}$ convierte vectores de $\R^n$ en vectores de $\R^m$.
\end{itemize}

\end{definition}

Para asegurar que comprendemos estas definiciones, realice los siguientes ejercicios:

\begin{enumerate}
    \item Considere la siguiente matriz como ejemplo
    $$ A = \begin{pmatrix} 1 & 2 & 3 \\ 4 & 5 & 6 \\ 7 & 8 & 9  \end{pmatrix} $$
    Encuentre:
    \begin{itemize}
        \item $a_{13}$
        \item $a_{31}$
        \item $a_{21}$
        \item $a_{12}$
        \item $A_{1  \bullet}$
        \item $A_{\bullet  1}$
        \item $A_{\bullet  3}$
        \item $A_{2  \bullet}$
    \end{itemize}
    \item Realice el siguiente cálculo de multiplicación matriz-vector:
    $$  \begin{pmatrix} 0 & 1 & 0 \\ 2 & 1 & 1 \\ 3 & 2 & 0  \end{pmatrix} \begin{pmatrix} 1 \\ 1 \\ 4  \end{pmatrix} $$
    \item \textbf{(Matrices elementales)} Realice los siguientes productos:
$$  \begin{pmatrix} 1 & 0 & 0 \\ 0 & 1 & 0 \\ 0 & 0 & 1  \end{pmatrix} \begin{pmatrix} a \\b \\c\end{pmatrix}$$
$$  \begin{pmatrix} 1 & 0 & \lambda \\ 0 & 1 & 0 \\ 0 & 0 & 1  \end{pmatrix} \begin{pmatrix} a \\ b \\ c \end{pmatrix} $$
$$  \begin{pmatrix} 0 & 1 \\ 1 & 0  \end{pmatrix} \begin{pmatrix} a \\ b \end{pmatrix}$$
Describa en sus palabras cómo \textit{actúa} la matriz sobre el vector en cada caso. Matrices de este tipo van a ser importantes a la hora de estudiar los sistemas de ecuaciones lineales.
\end{enumerate}

\subsection{Problemas computacionales}

Ver notebook número 4.