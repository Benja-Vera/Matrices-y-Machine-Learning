\documentclass{book}
\usepackage[letterpaper, rmargin=5em, lmargin=5em, textheight=63em]{geometry}
\usepackage[utf8]{inputenc}
\usepackage[english,spanish]{babel}
\usepackage{csquotes}
\usepackage{fancyhdr}
\usepackage[dvipsnames]{xcolor}
\usepackage{amssymb}
\usepackage{amsmath}
\usepackage{amsthm}
\usepackage{fontawesome}
\usepackage[unicode]{hyperref}
\usepackage{blindtext}
\usepackage{multicol}
\usepackage{caption}
\usepackage{physics}
\usepackage{graphicx}
\usepackage{subfiles}
\usepackage[style=alphabetic]{biblatex}
\addbibresource{referencias.bib}
\usepackage{tcolorbox}
% hyperref
\hypersetup{
    colorlinks=true,
    linkcolor=blue,
    filecolor=magenta,  
    urlcolor=black,
}

% Nombre de table de contenidos
\addto\captionsspanish{
  \renewcommand{\contentsname}
    {Contenidos}
}

\begin{document}

% Formato de teoremas y cositas
\theoremstyle{plain}
\newtheorem{theorem}{Teorema}[chapter]
\theoremstyle{definition}
\newtheorem{definition}[theorem]{Definición}[chapter]
\theoremstyle{plain}
\newtheorem{proposition}[theorem]{Proposición}[chapter]
\theoremstyle{plain}
\newtheorem{corollary}[theorem]{Corolario}[chapter]
\theoremstyle{definition}
\newtheorem{example}[theorem]{Ejemplo}[chapter]
\theoremstyle{plain}
\newtheorem{lemma}[theorem]{Lema}[chapter]
\theoremstyle{remark}
\newtheorem*{remark}{Nota}
\theoremstyle{remark}
\newtheorem*{question}{Pregunta}

% configuración adicional de teoremas
\renewcommand\qedsymbol{$\blacksquare$}

\begin{titlepage}
    \begin{center}
        \vspace*{1cm}
            
        \Huge
        \textbf{Matrices y Machine Learning}
            
        \vspace{0.5cm}
        \LARGE
        Apuntes y Ejercicios
            
        \vspace{2em}
            
        \textbf{Benja Vera}

        \vspace{2em}

        \includegraphics*[width=0.2\textwidth]{img/I1.png}
            
        \vfill
        
        \Huge
        \[f(x) = \sigma(Wx + b)\]

        \vfill
            
        \Large
        Versión actualizada al\\
        \today
        
        \vspace{3cm}
    \end{center}
\end{titlepage}
\tableofcontents

\chapter{Introducción}
\subfile{1-introducción.tex}

\chapter{Cálculo en una Variable}
\subfile{2-svc.tex}

\chapter{Álgebra Lineal}
\subfile{3-lineal.tex}

\printbibliography[heading=bibintoc]

\appendix

\chapter{Consideraciones Topológicas}
\subfile{A-topología.tex}

\end{document}