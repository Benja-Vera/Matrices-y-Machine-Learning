\documentclass{book}
\usepackage[letterpaper, rmargin=5em, lmargin=5em, textheight=63em]{geometry}
\usepackage[utf8]{inputenc}
\usepackage[english,spanish]{babel}
\usepackage{csquotes}
\usepackage{fancyhdr}
\usepackage[dvipsnames]{xcolor}
\usepackage{amssymb}
\usepackage{amsmath}
\usepackage{amsthm}
\usepackage{fontawesome}
\usepackage[unicode]{hyperref}
\usepackage{blindtext}
\usepackage{multicol}
\usepackage{caption}
\usepackage{physics}
\usepackage{graphicx}
\usepackage{subfiles}
\usepackage[style=alphabetic]{biblatex}
\addbibresource{referencias.bib}
\usepackage{tcolorbox}
% Alfabeto
\newcommand{\A}{\mathcal{A}}
\newcommand{\B}{\mathcal{B}}
\newcommand{\C}{\mathbb{C}}
\newcommand{\F}{\mathcal{F}}
\newcommand{\I}{\mathcal{I}}
\newcommand{\K}{\mathcal{K}}
\renewcommand{\L}{\mathcal{L}}
\newcommand{\M}{\mathcal{M}}
\newcommand{\N}{\mathbb{N}}
\renewcommand{\P}{\mathcal{P}}
\newcommand{\Q}{\mathbb{Q}}
\newcommand{\R}{\mathbb{R}}
\renewcommand{\S}{\mathcal{S}}
\newcommand{\T}{\mathcal{T}}
\newcommand{\Z}{\mathbb{Z}}

% Griego
\renewcommand{\epsilon}{\varepsilon}
\renewcommand{\emptyset}{\varnothing}

% GEOMETRY
\setlength{\headheight}{14pt}
\setlength{\parskip}{1em}
\pagestyle{fancy}
\renewcommand{\chaptermark}[1]{\markboth{#1}{#1}}
\cfoot{\thepage}
\fancyhead[L]{Matrices y Machine Learning}
\fancyhead[R]{\leftmark}

% TIKZ STUFF
\definecolor{dBlue}{RGB}{62, 71, 151}
\definecolor{lGreen}{RGB}{137, 190, 111}
\definecolor{lBlue}{RGB}{140, 208, 242}
\definecolor{orange}{RGB}{239, 147, 85}
\definecolor{black}{RGB}{0, 0, 0}
\definecolor{red}{RGB}{176, 52, 52}
\definecolor{grey}{RGB}{128, 128, 128}
\definecolor{dGreen}{RGB}{0, 140, 0}
\definecolor{purple}{RGB}{125, 0, 170}
\definecolor{pink}{RGB}{173, 109, 168}
\definecolor{lavander}{RGB}{101, 94, 163}
\definecolor{cyan}{RGB}{0, 157, 169}


% hyperref
\hypersetup{
    colorlinks=true,
    linkcolor=blue,
    filecolor=magenta,  
    urlcolor=magenta,
    citecolor=dGreen,
}

% Nombre de tabla de contenidos
\addto\captionsspanish{
  \renewcommand{\contentsname}
    {Contenidos}
}

% Operadores
\DeclareMathOperator{\Int}{int}
\DeclareMathOperator{\adh}{adh}

% Formato de teoremas y cositas
\theoremstyle{definition}
\newtheorem{definition}{Definición}[chapter]
\theoremstyle{plain}
\newtheorem{theorem}{Teorema}[chapter]
\theoremstyle{plain}
\newtheorem{proposition}{Proposición}[chapter]
\theoremstyle{plain}
\newtheorem{corollary}{Corolario}[chapter]
\theoremstyle{definition}
\newtheorem{example}{Ejemplo}[chapter]
\theoremstyle{plain}
\newtheorem{lemma}{Lema}[chapter]
\theoremstyle{remark}
\newtheorem*{remark}{Nota}
\theoremstyle{remark}
\newtheorem*{question}{Pregunta}

% configuración adicional de teoremas
\renewcommand\qedsymbol{$\blacksquare$}

\begin{document}

% Formato de teoremas y cositas
\theoremstyle{plain}
\newtheorem{theorem}{Teorema}[chapter]
\theoremstyle{definition}
\newtheorem{definition}[theorem]{Definición}[chapter]
\theoremstyle{plain}
\newtheorem{proposition}[theorem]{Proposición}[chapter]
\theoremstyle{plain}
\newtheorem{corollary}[theorem]{Corolario}[chapter]
\theoremstyle{definition}
\newtheorem{example}[theorem]{Ejemplo}[chapter]
\theoremstyle{plain}
\newtheorem{lemma}[theorem]{Lema}[chapter]
\theoremstyle{remark}
\newtheorem*{remark}{Nota}
\theoremstyle{remark}
\newtheorem*{question}{Pregunta}

% configuración adicional de teoremas
\renewcommand\qedsymbol{$\blacksquare$}

\begin{titlepage}
    \begin{center}
        \vspace*{1cm}
            
        \Huge
        \textbf{Ecuaciones Diferenciales}
            
        \vspace{0.5cm}
        \LARGE
        Problemas Resueltos
            
        \vspace{1.5cm}
            
        \textbf{Benja Vera}
            
        \vfill
            
        Escrito para acompañar el curso de\\
        MA2601 Ecuaciones Diferenciales Ordinarias
            
        \vspace{0.8cm}
            
        \includegraphics[width=0.4\textwidth]{img/I1.png}
            
        \Large
        Facultad de Ciencias Físicas y Matemáticas\\
        Universidad de Chile\\
        \today
            
    \end{center}
\end{titlepage}
\tableofcontents

\chapter{Introducción}
\subfile{1-introducción.tex}

\chapter{Cálculo en una Variable}
\subfile{2-svc.tex}

\chapter{Álgebra Lineal}
\subfile{3-lineal.tex}

\printbibliography[heading=bibintoc]

\appendix

\chapter{Consideraciones Topológicas}
\subfile{A-topología.tex}

\end{document}