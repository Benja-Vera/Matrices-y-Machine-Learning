En este primer capítulo, se resumen los elementos principales del cálculo en una variable que son necesarios para desarrollar la teoría de la optimización. Los dos algoritmos principales de minimización que se exploran son el \textbf{método de descenso} y el \textbf{método de Newton}. Siendo el primero sin duda el más utilizado en el mundo del machine learning hoy en día, aunque algunos autores han sugerido utilizar variaciones del otro (ver por ejemplo \cite{optimizacion-methods-for-dl}). En la sección \ref{svc:sec:derivada} se desarrolla el concepto intuitivo de la derivada de una función, junto con un resumen de las propiedades que esperaríamos que esta cumpliera. Luego, en la sección \ref{svc:sec:reglas} hablamos más bien de cómo realmente calcular la derivada de una función, estas ideas se enlazan con la sección anterior a través de un ejemplo de cómo las primeras dos derivadas de una función se pueden utilizar para esbozar su gráfico completo, el capítulo termina con una discusión sobre la \textbf{regla de la cadena}, concepto fundamental para lo que sigue. Después de esto, la sección \ref{svc:sec:taylor} aplica lo discutido sobre la regla de la cadena para construir los llamados \textbf{polinomios de Taylor} asociados a una función, los cuales vamos interpretar como buenas aproximaciones de la función cerca de un cierto punto $x_0$. El capítulo termina en la sección \ref{svc:sec:opti}, la cual expone los dos algoritmos principales de optimización mencionados al inicio.

Las ideas mencionadas en este capítulo se encuentran más detalladas, por ejemplo, en \cite{apunte-mate1}. Pero para una introducción completamente rigurosa al cálculo de una variable, se recomienda ver lo expuesto en \cite{spivak1988cálculo}

\section{La derivada y sus propiedades} \label{svc:sec:derivada}

\section{Las reglas de derivación} \label{svc:sec:reglas}

\section{Polinomios de Taylor} \label{svc:sec:taylor}

\section{Algoritmos de optimización} \label{svc:sec:opti}

\section{Ejercicios}

\subsection{Preliminares}

\begin{enumerate}

    \item Analice, mediante una tabla, los signos de las siguientes expresiones ya factorizadas

    \begin{enumerate}
        \item $\frac{x + 1}{x - 1}$
        \item $(x + 2)(x - 3)$
        \item $(x + 4)(1 - x)$
        \item $\frac{(x + 1)(x - 1)}{x-3}$
    \end{enumerate}

    \item Factorice las siguientes expresiones y luego analice sus signos como en el item anterior
    
    \begin{enumerate}
        \item $\frac{x^2 - 4}{x + 1}$
        \item $x^2 - 4x + 3$
        \item $x^2 - 3x$
    \end{enumerate}

\end{enumerate}

\subsection{Cálculo de derivadas por definición}

Entre las dos definiciones de derivada que hemos visto:
\[f'(x) = \lim_{h \to 0} \frac{f(x + h) - f(x)}{h}, \qquad f'(x) = \lim_{w \to x} \frac{f(w) - f(x)}{w - x}\]
Utilice la que más le convenga para calcular las derivadas de las siguientes funciones

\begin{enumerate}
    \item $f(x) = \frac{1}{x}$
    \item $f(x) = \sqrt{x}$
    \item $f(x) = x^3$
\end{enumerate}

\subsection{Reglas de derivación sin regla de la cadena}

Utilizando las reglas de derivación, calcule las derivadas de las siguientes funciones
\begin{multicols}{2}
    \begin{enumerate}
        \item $f(x) = x^4$
        \item $f(x) = 3x^5 - x^3$
        \item $f(x) = 4x^2 - 3\pi^2$
        \item $f(x) = (x+1)(x-1)$
        \item $f(x) = \frac{x+1}{x-1}$
        \item $f(x) = \frac{1}{\sqrt{x}}$
    \end{enumerate}
\end{multicols}

\subsection{La regla de la cadena}

Utilizando las reglas de derivación conocidas, además de la regla de la cadena, obtenga las derivadas de las siguientes funciones.
\begin{multicols}{3}
    \begin{enumerate}
        \item $f(x) = (1 + \sqrt{x})^2$
        \item $f(x) = \sqrt[3]{2x}$
        \item $f(x) = (4 + 2x)^{2023}$
        \item $f(x) = \sqrt{1 - x^2}$
        \item $f(x) = (\frac{1}{x} + x)^3$
        \item $f(x) = (1 - x)^5$
    \end{enumerate}
\end{multicols}

\subsection{Problemas varios sobre rectas tangentes}
Para esta sección, recuerde que la recta tangente al gráfico de una función $f$ en el punto $(x_0, f(x_0))$ viene dada por la ecuación
\[y = f(x_0) + f'(x_0)(x - x_0)\]
Además, recordemos también el hecho de que si $L_1$ y $L_2$ son dos rectas dadas respectivamente por
\[y = m_1 x + n_1, \qquad y = m_2 x + n_2\]
Entonces estas rectas son perpendiculares cuando $m_1 \cdot m_2 = -1$

\begin{enumerate}
    \item Considere el gráfico de la función $f(x) = \frac{1}{x}$ y sea $x_0 > 0$ fijo.
    \begin{enumerate}
        \item Encuentre la ecuación de la recta tangente al gráfico de $f$ en el punto $(x_0, 1/x_0)$. Debería obtener una recta cuyos parámetros vienen escritos en términos de $x_0$.
        \item Obtenga los puntos de intersección entre la recta obtenida anteriormente y los ejes de coordenadas. Recuerde que esto se puede hacer imponiendo $x = 0$ y $y = 0$ según corresponda. Debería obtener dos puntos cuyas coordenadas vienen dadas en términos de $x_0$.
        \item Al dibujar lo que está sucediendo, notará que se forma un triángulo (dado por la recta tangente y los ejes de coordenadas). Calcule su área ¿Qué puede decir con respecto a este área?
    \end{enumerate}
    \item En este problema, vamos a probar el hecho de que las rectas tangentes a los círculos son siempre perpendiculares al radio. Hecho que ya se conoce de la geometría euclideana. Recordemos para esto que la ecuación que define a una circunferencia de radio $1$ centrada en el punto $(0, 0)$ viene dada por
    \[x^2 + y^2 = 1\]
    De modo que depejando $y$, la función que define al semicírculo superior de esta circunferencia viene dada por
    \[f(x) = \sqrt{1 - x^2}\]
    Dado $x_0 \in ]-1, 1[$ fijo, proceda como sigue:
    \begin{enumerate}
        \item Considere el radio que une el origen con el punto $(x_0, f(x_0))$. Conociendo dos puntos, calcule la pendiente de este radio.
        
        \textbf{Nota:} Es posible calcular también la ecuación de la recta que define a este radio, pero no la necesitamos realmente.

        \item Calcule además la derivada de la función $f$ en el punto $x_0$.
        \item Interpretando sus resultados anteriores, concluya lo pedido.
    \end{enumerate}
\end{enumerate}

\subsection{Máximos y mínimos}

\begin{enumerate}
    \item Pruebe, utilizando lo expuesto en este capítulo, el hecho ya conocido de que si $f(x) = ax^2 + bx + c$ es una función cuadrática cualquiera con $a \neq 0$, entonces el punto $x_0 = \frac{-b}{2a}$ corresponde a un mínimo si $a > 0$ y un máximo si $a < 0$.
    \item Se desea encontrar el punto $P = \qty(x, \frac{1}{x})$ con $x > 0$ (construido así de modo que $P$ pertenece al gráfico de la función $f(x) = 1/x$) cuya distancia euclideana al origen de coordenadas sea mínima. Para esto, y teniendo en cuenta la siguiente figura, proceda como sigue:
    
    [FIGURA]
    \begin{enumerate}
        \item Pruebe mediante el teorema de pitágoras que para $x > 0$ fijo, la distancia viene dada por
        \[d(x) = \sqrt{x^2 + \frac{1}{x^2}}\]
        \item Obtenga la derivada de esta función $d'(x)$ y resuelva la ecuación $d'(x) = 0$. Llamemos $x_0$ al valor obtenido.
        \item Con este valor en mano, calcule $d''(x_0)$ y concluya.
    \end{enumerate}
\end{enumerate}

\subsection{Gráficos de funciones}

\subsection{Polinomios de Taylor}